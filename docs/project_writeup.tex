\documentclass[letterpaper,twocolumn]{article}
\usepackage{graphicx}
\usepackage{amsfonts}
\usepackage{mathtools}
\usepackage{mathrsfs}
\usepackage{fancyhdr} % needed just for the "name on every page header, below paygestyle
\usepackage[margin=1in]{geometry} % should institute 0.5in margins, untested
% investigate L/R margins vs. top, def a diff
\usepackage[textfont={scriptsize},font=bf]{caption}
\usepackage{color}
\usepackage{hyperref}
\setlength{\columnsep}{0.75cm}

\begin{document}
\title{MA 703, Project Report}
\author{Austin Soplata}
\maketitle
\pagestyle{fancy} % needed just for the header
\fancyhead[L]{{\it Austin Soplata, \the\year}}
\fancyfoot[R]{{\it\small Proceedings of the National Academy of Austin Soplata}}
% -----------------------------------------------------------------------------
\section{Proposal}

\indent\indent derp

derp \footnote{Data can be found here {\color{blue} \href{http://www-personal.umich.edu/~mejn/netdata/}{Data page}} }

\begin{equation}
C_m \frac{d V_m (t)}{dt} = I(t) - \frac{V_m (t)}{R_m}
\end{equation}

\begin{figure}
\centering
% \graphicspath{ {./figures/} }
\includegraphics[scale=0.37]{fig01.png}
\caption{The second equation on page 254 of {\it H. Drees, L. de Haan, and S. Resnick, “How to make a Hill plot,” The Annals of Statistics, vol. 28, no. 1, pp. 254–274, 2000.}, which shows no extra 1 term.}
\end{figure}






% % ----------------------------------------------------------------------------
% \pagebreak
% \onecolumn
% \appendix
% \label{a1}
% \section{Appendix: {\sc R} code.}
% \begin{verbatim}
% \end{verbatim}
\end{document}
